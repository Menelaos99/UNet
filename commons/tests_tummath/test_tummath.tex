\documentclass[DIV=13]{scrartcl}

\usepackage{tumbase}
\usepackage{tumcolors}
\usepackage{tumlang}
\usepackage[font=@TEXTFONT@]{tumfonts}
\usepackage{tumboxes}
\tcbuselibrary{listings}
\usepackage{tummath}

\usepackage{multicol}
\usepackage[hidelinks]{hyperref}


\title{The \texttt{tummath} package}
\author{\LaTeX4EI}
\date{17.06.2021}


\makeatletter
\@ifpackagelater{siunitx}{2021/05/01}{
  %%% siunitx v3
}{
  %%% siunitx v2
  % fake compatibility with v3 to make this testcase compile
  % with all versions
  \newcommand{\complexnum}[2][]{\num[#1]{#2}}
  \newcommand{\complexqty}[2][]{\SI[#1]{#2}}
}
\makeatother

\DeclareSIUnit{\statV}{statV}


\begin{document}
\maketitle

\begin{abstract}
  The aim of the \texttt{tummath} package is to provide macros for consistent
  mathematical typesetting. The package adds shorthand macros for frequently
  used constructs and implements symbols and operators missing in the default
  \LaTeX\ environment to provide a more complete set of symbols.
  Whereever possible, we follow the ISO \num{80000}-x norms (mainly part 2 of
  the series is of importance here).

  This document introduces all macros provided by the \texttt{tummath}
  package and in addition to that, it serves as a cheat-sheet for available
  \LaTeX\ symbols. We organized the symbols into mathmatical topics and tried
  to include the most relevant symbols.

  (Disclaimer: The provided examples should give a visual impression of the
  symbols. They do not necessarily make sense mathematically.)
\end{abstract}

\begin{multicols*}{2}
  \tableofcontents
\end{multicols*}


\section{Package dependencies}
The \texttt{tummath} package depends on the following list of \LaTeX\
packages. It was designed to work with the fonts provided by the
\texttt{tumfonts} package.
\begin{multicols}{4}
  \begin{itemize}
    \item etoolbox
    \item array
    \item booktabs
    \item mathtools
    \item scalerel
    \item bm
    \item tensor
    \item siunitx
    \item tikz
  \end{itemize}
\end{multicols}


\section{Basic Math notation}
\subsection{Logic}
This package adds macros for logical implications and equivalence to provide
a full set of logical operators.
% \begin{noindent}
\begin{center}
  \begin{tabular}{lll}
    \toprule
    \textbf{Description} & \textbf{Macro}      & \textbf{Output} \\
    \midrule
    and                  & \verb|A \land B|    & $A \land B$     \\
    or                   & \verb|A \lor B|     & $A \lor B$      \\
    not                  & \verb|\lnot A|      & $\lnot A$       \\
    implication          & \verb|A \lifthen B| & $A \lifthen B$  \\
    equivalence          & \verb|A \lequiv B|  & $A \lequiv B$   \\
    \midrule
    allquantor           & \verb|\forall x|    & $\forall x$     \\
    existence quantor    & \verb|\exists x|    & $\exists x$     \\
    \bottomrule
  \end{tabular}
\end{center}
% \end{noindent}
\begin{gather}
  \lnot(\forall x \in A : S(x)) \lequiv \exists x \in A : \lnot S(x), \\
  (A \lequiv B) \lequiv ((A \lifthen B) \land (B \lifthen A)), \\
  (A \lifthen B) \lequiv \lnot(A \land \lnot B), \\
  (A \lor B) \lequiv \lnot(\lnot A \land \lnot B).
\end{gather}


\subsection{Set Theory}
The \verb|\set{}| macro is implemented with \texttt{mathtools}
paired delimiters and can take a size macro as optional argument.
Both the set braces and the vertical bar added by the \verb|\given| macro
will scale according to the size macro. The starred version of the command
\verb|\set*{}| will scale according to the content using the \verb|\left|,
\verb|\right|, \verb|\middle| commands internally.
% \begin{noindent}
\begin{center}
  \begin{tabular}{lll}
    \toprule
    \textbf{Description} & \textbf{Macro}          & \textbf{Output}    \\
    \midrule
    set                  & \verb|\set{1,2,3,4}|    & $\set{1,2,3,4}$    \\
    set with condition   & \verb|\set{x \given A}| & $\set{x \given A}$ \\
    empty set            & \verb|\emptyset,\set{}| & $\emptyset,\set{}$ \\
    \midrule
    union                & \verb|A \cup B|         & $A \cup B$         \\
    intersection         & \verb|A \cap B|         & $A \cap B$         \\
    set difference       & \verb|A \setminus B|    & $A \setminus B$    \\
    subset               & \verb|A \subseteq B|    & $A \subseteq B$    \\
                         & \verb|B \supseteq A|    & $B \supseteq A$    \\
    proper subset        & \verb|A \subset B|      & $A \subset B$      \\
                         & \verb|B \supset A|      & $B \supset A$      \\
    not subset           & \verb|A \nsubseteq B|   & $A \nsubseteq B$   \\
                         & \verb|B \nsupseteq A|   & $B \nsupseteq A$   \\
    element of           & \verb|x \in A|          & $x \in A$          \\
    not element of       & \verb|x \notin A|       & $x \notin A$       \\
    \midrule
    cardinality          & \verb|\card A|          & $\card A$          \\
    supremum             & \verb|\sup A|           & $\sup A$           \\
    infimum              & \verb|\inf A|           & $\inf A$           \\
    maximum              & \verb|\max A|           & $\max A$           \\
    minimum              & \verb|\min A|           & $\min A$           \\
    \bottomrule
  \end{tabular}
\end{center}
% \end{noindent}
\begin{gather}
  \emptyset\subseteq M \quad\text{and}\quad M\subseteq M, \\
  A=B \lequiv ((A \subseteq B)\land(B \subseteq A)), \\
  A \cap B = \set*{x \given x\in A \land x\in B}, \\
  A \subseteq B \lequiv A \cap B = A, \\
  \mathcal{P}(M) = \set{A \given A\subseteq M}, \\
  A\times B = \set[\Big]{(a,b) \given a\in A, b\in B}, \\
  \card \Rnums = \infty, \\
  \max A = \sup A \in A, \\
  \min A = \inf A \in A.
\end{gather}


\subsection{Maps and Functions}
% \begin{noindent}
\begin{center}
  \begin{tabular}{lll}
    \toprule
    \textbf{Description} & \textbf{Macro}   & \textbf{Output} \\
    \midrule
    mapping domains      & \verb|X \to Y|   & $X \to Y$       \\
    mapping elements     & \verb|x \mapsto \function{f}{x}|
                         & $x \mapsto \function{f}{x}$        \\
    image                & \verb|\im(X)|    & $\im(X)$        \\
    pre-image            & \verb|\preim(Y)| & $\preim(Y)$     \\
    composition          & \verb|g \circ f| & $g \circ f$     \\
    identity             & \verb|\id_X|     & $\id_X$         \\
    \bottomrule
  \end{tabular}
\end{center}
% \end{noindent}
\begin{gather}
  f : \begin{cases}
    \Rnums \to \Rnums_+ \\
    x \mapsto \e^x
  \end{cases}, \\
  f : M \to N, \\
  \im_f(A) \coloneq \set*{f(a) \in N \given a \in A \subseteq M}, \\
  \preim_f(B) \coloneq
  \set*{m \in M \given \function{f}{m} \in B \subseteq N}, \\
  g : N \to P, \\
  g \circ f : \begin{cases}
    M \to P \\
    m \mapsto (g \circ f)(m) \coloneq \function{g}{\function{f}{m}}
  \end{cases}, \\
  g : N \to M, \\
  g \circ f = \id_M \;\text{and}\; f \circ g = \id_N
  \;\lifthen\; g=f^{-1}.
\end{gather}

This package relies heavily on the paired delimiter framework introduced by
the \texttt{mathtools} package. This gives the user the ability to
scale the delimiters manually by adding the size command as an optional
argument or use the starred version for automatic scaling, which uses the
\verb|\left| and \verb|\right| macros internally.
% \begin{noindent}
\begin{center}
  \begin{tabular}{lll}
    \toprule
    \textbf{Size} & \textbf{Macro}
                  & \textbf{Output}               \\
    \midrule
    normal        & \verb|\function{f}{x}|
                  & $\function{f}{x}$             \\
    big           & \verb|\function[\big]{f}{x}|
                  & $\function[\big]{f}{x}$       \\
    Big           & \verb|\function[\Big]{f}{x}|
                  & $\function[\Big]{f}{x}$       \\
    bigg          & \verb|\function[\bigg]{f}{x}|
                  & $\function[\bigg]{f}{x}$      \\
    Bigg          & \verb|\function[\Bigg]{f}{x}|
                  & $\function[\Bigg]{f}{x}$      \\
    automatic     & \verb|\function*{f}{x}|
                  & $\function*{f}{x}$            \\
    \bottomrule
  \end{tabular}
\end{center}
% \end{noindent}


\subsection{Numbers}
For convenience, shorthand macros for the basic number sets are defined.
% \begin{noindent}
\begin{center}
  \begin{tabular}{lll}
    \toprule
    \textbf{Description} & \textbf{Macro}    & \textbf{Output} \\
    \midrule
    natural numbers      & \verb|\Nnums|     & $\Nnums$        \\
    integer numbers      & \verb|\Znums|     & $\Znums$        \\
    rational numbers     & \verb|\Qnums|     & $\Qnums$        \\
    real numbers         & \verb|\Rnums|     & $\Rnums$        \\
    complex numbers      & \verb|\Cnums|     & $\Cnums$        \\
    quaternions          & \verb|\Hnums|     & $\Hnums$        \\
    \bottomrule
  \end{tabular}
\end{center}
% \end{noindent}

According to the ISO \num{80000} the symbols for imaginary units should be
printed in upright math font.
The accent macro \verb|\k| is saved as \verb|\ogonek|. The text macros
\verb|\i| and \verb|\j| for dotless versions of the letters i and j are saved
as \verb|\dotlessi| and \verb|\dotlessj|.

The shorthand macro for the default imaginary unit \verb|\imu| defaults to
\verb|\j| (engineering notation) and may be redefined by the user in the
document preamble.

The macros for real and imaginary part as well as the macro for the absolute
value are implemented using \texttt{mathtools} paired delimiters and can be
scaled with an optional argument or the starred version.
% \begin{noindent}
\begin{center}
  \begin{tabular}{lll}
    \toprule
    \textbf{Description} & \textbf{Macro}             & \textbf{Output}     \\
    \midrule
    imaginary units      & \verb|\i, \j, \k|          & $\i, \j, \k$        \\
    default imag. unit   & \verb|\imu|                & $\imu$              \\
    complex numbers      & \verb|z = a + \i b|        & $z = a + \i b$      \\
    complex conjugate    & \verb|\cc{z} = a - \i b|   & $\cc{z} = a - \i b$ \\
    real part            & \verb|\Re{z} = a|          & $\Re{z} = a$        \\
    imaginary part       & \verb|\Im{z} = b|          & $\Im{z} = b$        \\
    magnitude            & \verb|r = \abs{z}|         & $r = \abs{z}$       \\
    argument             & \verb|\varphi = \arg(z)|   & $\varphi = \arg(z)$ \\
    euler representation & \verb|z = r\e^{\i\varphi}|
                         & $z = r\e^{\i\varphi}$                            \\
    \midrule
    quaternions          & \verb|q = a + \i b + \j c + \k d|
                         & $q = a + \i b + \j c + \k d$                     \\
    \bottomrule
  \end{tabular}
\end{center}
% \end{noindent}

According to the ISO \num{80000} there is no special notation for complex
quantities. Like for any real quantity, italic letters are used to denote
complex quantities. In some fields of electrical engineering, it is common
practice to underline complex quantities. This is allowed according to the
German DIN EN \num{60027}-1. To provide this notation, we implement the
macro \verb|\cnum{z}| printing the complex quantity $\cnum{z}$. Similarly,
the macros \verb|\cvec{v}| and \verb|\cmat{M}| are implemented to denote
complex valued vectors $\cvec{v}$ and matrices $\cmat{M}$.

If you choose to use this notation, you should redefine the macro for the
complex conjugate in your preamble to use the star notation.
\begin{verbatim}
  \renewcommand*{\cc}[1]{{#1^\ast}}
\end{verbatim}


\subsection{Symbols for relations}
\LaTeX\ provides many symbols for relations. The table summarizes a subset
of symbols that is recommended by the ISO \num{80000}.
This package adds symbols for definitions $\coloneq$, $\eqcolon$, $\defeq$,
correspondence $\correq$, and an equal sign with exclamation mark $\excleq$.
% \begin{noindent}
\begin{center}
  \begin{tabular}{lll|lll}
    \toprule
    \textbf{Description}   & \textbf{Macro}   & \textbf{Output} &
    \textbf{Description}   & \textbf{Macro}   & \textbf{Output} \\
    \midrule
    equal                  & \verb|=|         & $=$             &
    equivalent             & \verb|\equiv|    & $\equiv$        \\
    not equal              & \verb|\neq|      & $\neq$          &
    congruent / isomorphic & \verb|\cong|     & $\cong$         \\
    equal by definition    & \verb|\coloneq|  & $\coloneq$      &
    asymptotically equal   & \verb|\simeq|    & $\simeq$        \\
                           & \verb|\eqcolon|  & $\eqcolon$      &
    approximately equal    & \verb|\approx|   & $\approx$       \\
                           & \verb|\defeq|    & $\defeq$        &
    proportional           & \verb|\sim|      & $\sim$          \\
    corresponds to         & \verb|\correq|   & $\correq$       &
                           & \verb|\propto|   & $\propto$       \\
    shall be equal         & \verb|\excleq|   & $\excleq$       \\
    \midrule
    less than              & \verb|<|         & $<$             &
    greater than           & \verb|>|         & $>$             \\
    less or equal          & \verb|\leq|      & $\leq$          &
    greater or equal       & \verb|\geq|      & $\geq$          \\
    much less              & \verb|\ll|       & $\ll$           &
    much greater           & \verb|\gg|       & $\gg$           \\
    \bottomrule
  \end{tabular}
\end{center}
% \end{noindent}


\subsection{Intervals and rounding symbols}
The interval and rounding symbols are implemented using \texttt{mathtools}
paired delimiters and can be scaled using an optional size argument or by
using the starred version to automatically scale to the content.
% \begin{noindent}
\begin{center}
  \begin{tabular}{lll}
    \toprule
    \textbf{Description} & \textbf{Macro}          & \textbf{Output}    \\
    \midrule
    open interval        & \verb|\ointerval{a,b}|  & $\ointerval{a,b}$  \\
    closed interval      & \verb|\cinterval{a,b}|  & $\cinterval{a,b}$  \\
    semi-open (left)     & \verb|\ocinterval{a,b}| & $\ocinterval{a,b}$ \\
    semi-open (right)    & \verb|\cointerval{a,b}| & $\cointerval{a,b}$ \\
    \midrule
    floor function       & \verb|\floor{x}|        & $\floor{x}$        \\
    ceil function        & \verb|\ceil{x}|         & $\ceil{x}$         \\
    \bottomrule
  \end{tabular}
\end{center}
% \end{noindent}
\begin{gather}
  \cinterval{a,b} \coloneq \set{x \in \Rnums \given a \leq x \leq b}, \\
  \ointerval{a,b} \coloneq \set{x \in \Rnums \given a < x < b}, \\
  \ocinterval{a,b} \coloneq \set{x \in \Rnums \given a < x \leq b}, \\
  \cointerval{a,b} \coloneq \set{x \in \Rnums \given a \leq x < b}, \\
  \floor{2.8} = 2, \\
  \ceil{2.4} = 3, \\
\end{gather}


\section{Analysis}
\subsection{General symbols}
According to the ISO \num{80000}, special constants such as the euler number
$\e$ have to be printed in upright math font. The macros for the absolute
value and the Landau symbol are implemented using \texttt{mathtools}
paired delimiters.
% \begin{noindent}
\begin{center}
  \begin{tabular}{lll}
    \toprule
    \textbf{Description}  & \textbf{Macro}   & \textbf{Output}     \\
    \midrule
    absolute value        & \verb|\abs{x}|   & $\abs{x}$           \\
    signum                & \verb|\sgn(x)|   & $\sgn(x)$           \\
    \midrule
    euler number          & \verb|\e|        & $\e$                \\
    exponential function  & \verb|\exp(x)|   & $\exp(x)$           \\
    \midrule
    limes                 & \verb|\lim_{n\to\infty} a_n|
                          & $\lim_{n\to\infty} a_n$                \\
    limes superior        & \verb|\limsup_{n\to\infty} a_n|
                          & $\limsup_{n\to\infty} a_n$             \\
    limes inferior        & \verb|\liminf_{n\to\infty} a_n|
                          & $\liminf_{n\to\infty} a_n$             \\
    \midrule
    Landau symbol (big O) & \verb|\landau*{n^2}| & $\landau*{n^2}$ \\
    \bottomrule
  \end{tabular}
\end{center}
% \end{noindent}
\begin{gather}
  \abs{x}\coloneq\begin{cases}
    \phantom{-}x & x\geq 0 \\
    -x           & x<0
  \end{cases}, \qquad
  \sgn(x)\coloneq\begin{cases}
    \phantom{-}1 & x>0 \\
    \phantom{-}0 & x=0 \\
    -1           & x<0
  \end{cases}, \\
  \abs*{\frac{(a+b)c}{d-e}} + f\cdot\abs[\big]{g}, \\
  \abs{x+y} \leq \abs{x}+\abs{y}, \\
  \e = \e^1 \coloneq \exp(1) = \sum_{k=0}^\infty \frac{1}{k!},\\
  \lim_{n\to\infty}(a_n) = a \quad\lequiv\quad
  \forall\epsilon>0\;\exists N\in\Nnums\;\forall n\geq N:
  \abs{a_n-a}<\epsilon,\\
  \limsup_{n\to\infty} a_n \coloneq \sup H(a_n), \\
  \liminf_{n\to\infty} a_n \coloneq \inf H(a_n), \\
  n! = \sqrt{2\pi n}\left(\frac{n}{\e}\right)^n
  \left(1+\landau*{\frac{1}{n}}\right) \quad\text{for } n\to\infty, \\
  f\in\landau*{n\log n}.
\end{gather}


\subsection{Differentials}
According to the ISO \num{80000}, the differential symbol $\diff$ must be
printed in upright math font. The shorthand macros for derivatives and
partial derivatives typeset a fraction with differential symbols.
The optional argument specifies the order of the derivative and will print
the exponents accordingly.
% \begin{noindent}
\begin{center}
  \renewcommand{\arraystretch}{1.25}
  \begin{tabular}{lll}
    \toprule
    \textbf{Description} & \textbf{Macro}            & \textbf{Output}      \\
    \midrule
    differential element & \verb|\diff x|            & $\diff x$            \\
    derivative           & \verb|\dd{\vec{r}}{t}|    & $\dd{\vec{r}}{t}$    \\
    second derivative    & \verb|\dd[2]{\vec{r}}{t}| & $\dd[2]{\vec{r}}{t}$ \\
    $n$-th derivative    & \verb|\dd[n]{}{t}f(t)|    & $\dd[n]{}{t}f(t)$    \\
    \midrule
    partial derivatives  & \verb|\pdd[2]{}{t}\Psi(x,t)|
                         & $\pdd[2]{}{t}\Psi(x,t)$                          \\
    \bottomrule
  \end{tabular}
\end{center}
% \end{noindent}

\begin{gather}
  \function{f^\prime}{x} = \dd{\function{f}{x}}{x} = \lim_{x\to x_0}
  \frac{\function{f}{x}-\function{f}{x_0}}{x-x_0}, \\
  \function{f^{(n)}}{x} = \dd[n]{\function{f}{x}}{x}
  = \dd{}{x}\function{f^{(n-1)}}{x}, \\
  \diff H = \sum_{i=1}^{S}\left(\pdd{H}{p_i}\diff p_i +
    \pdd{H}{q_i}\diff q_i\right) + \pdd{H}{t}\diff t, \\
  \i\hbar\pdd{}{t}\Psi(x,t) =
  \left[-\frac{\hbar^2}{2m}\pdd[2]{}{x} + V(x)\right]\Psi(x,t).
\end{gather}


\subsection{Integrals}
% \begin{noindent}
\begin{center}
  \begin{tabular}{llc}
    \toprule
    \textbf{Description}    & \textbf{Macro}   & \textbf{Output} \\
    \midrule
    integral                & \verb|\int|      & $\int$          \\
    surface integral        & \verb|\iint|     & $\iint$         \\
    volume integral         & \verb|\iiint|    & $\iiint$        \\
    dots integral           & \verb|\idotsint| & $\idotsint$     \\
    closed integral         & \verb|\oint|     & $\oint$         \\
    closed surface integral & \verb|\oiint|    & $\oiint$        \\
    % closed volume integral & \verb|\oiiint|    & $\oiiint$       \\
    % sumint                 & \verb|\sumint|    & $\sumint$       \\
    \bottomrule
  \end{tabular}
\end{center}
% \end{noindent}
\begin{gather}
  \function{f}{z_0} = \frac{1}{2\pi\i}\oint_C
  \frac{\function{f}{z}\diff z}{z-z_0}, \\
  \oint_{\partial V} \vec{E}\cdot\diff\vec{A}
  = \int_V \div\vec{E}(\vec{r})\diff^3 r, \\
  \oint_{\partial A}\vec{a}\diff\vec{r}
  = \int_A \rot\vec{a}\cdot\diff\vec{A}, \\[1ex]
  \oiint_{\partial V}\vec{D}\diff\vec{A} = \iiint_V \rho\diff V, \\
  \oiint_{\partial V}\vec{B}\diff\vec{A} = 0, \\
  \oint_{\partial A}\vec{E}\diff\vec{s}
  = -\iint_A \pdd{\vec{B}}{t}\cdot\diff\vec{A}, \\
  \oint_{\partial A}\vec{H}\diff{\vec{s}}
  = \iint_A\left(\vec{j} + \pdd{\vec{D}}{t}\right)\cdot\diff\vec{A}.
\end{gather}


\section{Linear Algebra}
\subsection{Environments to print matrices}
The \verb|amsmath| package provides several environments to typeset matrices
and general arrays. This package adds an optional argument to
all matrix environments to configure the column alignment and/or add
vertical lines between columns.

By default the columns are center aligned, but in some cases it might be
beneficial to left or right align the matrix elements. Printing vertical
lines can be useful to typeset linear systems.

To provide horizontal lines with correct spacing, we redefined the
\verb|\midrule| macro from \texttt{booktabs} within matrix environments.

\begin{tcblisting}{TUMBox, sidebyside,
    title={Matrix without delimiters (default, centered alignment)}}
  \begin{equation}
    \begin{matrix}
      1   & 2  & 3  & -1 & 0  \\
      0   & -5 & 12 & 0  & -3 \\
      -10 & -3 & 1  & 4  & 7  \\
    \end{matrix}
  \end{equation}
\end{tcblisting}
\begin{tcblisting}{TUMBox, sidebyside,
    title=Matrix with parentheses (using right alignment)}
  \begin{equation}
    \begin{pmatrix}[*5r]
      1   & 2  & 3  & -1 & 0  \\
      0   & -5 & 12 & 0  & -3 \\
      -10 & -3 & 1  & 4  & 7  \\
    \end{pmatrix}
  \end{equation}
\end{tcblisting}
\begin{tcblisting}{TUMBox, sidebyside,
    title=Matrix with brackets (using left alignment)}
  \begin{equation}
    \begin{bmatrix}[*5l]
      1   & 2  & 3  & -1 & 0  \\
      0   & -5 & 12 & 0  & -3 \\
      -10 & -3 & 1  & 4  & 7  \\
    \end{bmatrix}
  \end{equation}
\end{tcblisting}
\begin{tcblisting}{TUMBox, sidebyside,
    title=Matrix with braces (using mixed alignment)}
  \begin{equation}
    \begin{Bmatrix}[l*3cr]
      1   & 2  & 3  & -1 & 0  \\
      0   & -5 & 12 & 0  & -3 \\
      -10 & -3 & 1  & 4  & 7  \\
    \end{Bmatrix}
  \end{equation}
\end{tcblisting}
\begin{tcblisting}{TUMBox, sidebyside,
    title=Matrix with vertical lines}
  \begin{equation}
    \begin{vmatrix}
      1   & 2  & 3  & -1 & 0  \\
      0   & -5 & 12 & 0  & -3 \\
      -10 & -3 & 1  & 4  & 7  \\
    \end{vmatrix}
  \end{equation}
\end{tcblisting}
\begin{tcblisting}{TUMBox, sidebyside,
    title=Matrix with double vertical lines}
  \begin{equation}
    \begin{Vmatrix}
      1   & 2  & 3  & -1 & 0  \\
      0   & -5 & 12 & 0  & -3 \\
      -10 & -3 & 1  & 4  & 7  \\
    \end{Vmatrix}
  \end{equation}
\end{tcblisting}
\begin{tcblisting}{TUMBox, sidebyside,
    title=Block matrix using horizontal and vertical lines,
    listing options={style=tcblatex, morekeywords={midrule}}
  }
  \begin{equation}
    \begin{pmatrix}[*3r|rr]
      1   & 2  & 3  & -1 & 0  \\
      0   & -5 & 12 & 0  & -3 \\
      \midrule
      -10 & -3 & 1  & 4  & 7  \\
    \end{pmatrix}
  \end{equation}
\end{tcblisting}
\begin{tcblisting}{TUMBox, sidebyside,
    title=A linear system of equations}
  \begin{equation}
    \begin{pmatrix}[*5c|c]
      0 & 0 & 1 & 3 & 3 & 2 \\
      1 & 2 & 1 & 4 & 3 & 3 \\
      1 & 2 & 2 & 7 & 6 & 5 \\
      2 & 4 & 1 & 5 & 3 & 4 \\
    \end{pmatrix}
  \end{equation}
\end{tcblisting}
\begin{tcblisting}{TUMBox, sidebyside,
    title=The Gauß-Algorithm to invert a matrix}
  \begin{gather}
    \begin{pmatrix}[*3r|*3r]
      1 & 0  & 1 & 1 & 0 & 0 \\
      2 & -1 & 3 & 0 & 1 & 0 \\
      1 & 2  & 0 & 0 & 0 & 1 \\
    \end{pmatrix}\\[2ex]
    \rightsquigarrow
    \begin{pmatrix}[*3r|*3r]
      1 & 0 & 0 & 6  & -2 & -1 \\
      0 & 1 & 0 & -3 & 1  & 1  \\
      0 & 0 & 1 & -5 & 2  & 1  \\
    \end{pmatrix}
  \end{gather}
\end{tcblisting}
% \begin{noindent}
\begin{tcblisting}{TUMBox, title=Fancy lines in a matrix,
    listing options={style=tcblatex, morekeywords={cmidrule}}
  }
  \begin{equation}
    \begin{pmatrix}
      0 & \multicolumn{1}{|c}{0} & 1 & 3 & 3 & 2 \\
      \cmidrule{2-3}
      1 & 2 & 1 & \multicolumn{1}{|c}{4} & 3 & 3 \\
      \cmidrule{4-5}
      1 & 2 & 2 & 7 & \multicolumn{1}{c|}{6} & 5 \\
      \cmidrule(r){6-6}
      2 & 4 & 1 & 5 & 3 & 4 \\
    \end{pmatrix}
  \end{equation}
\end{tcblisting}
% \end{noindent}


\subsection{Matrices}
Following the ISO \num{80000}, matrices should be printed in bold italic math
font. Macros for the most important matrix operations are provided.
The macros \verb|\Mat{}| and \verb|\Det{}| are shorthands for the
\texttt{bmatrix} and \texttt{vmatrix} environments when used in
\emph{display math} mode. When used in \emph{inline math} mode, the
\texttt{mathtools} environments \texttt{bsmallmatrix*} and
\texttt{vsmallmatrix*} are used. The optional arguments \verb|[l]|,
\verb|[c]|, or \verb|[r]| may be used to specify left, center, or right
alignment of the matrix columns.
% \begin{noindent}
\begin{center}
  \begin{tabular}{lll}
    \toprule
    \textbf{Description} & \textbf{Macro}              & \textbf{Output}    \\
    \midrule
    matrix               & \verb|\mat{A}|              & $\mat{A}$          \\
    transposed           & \verb|\mat{A}^\transp|      & $\mat{A}^\transp$  \\
    complex conjugate    & \verb|\cc{\mat{A}}|         & $\cc{\mat{A}}$     \\
    adjoint              & \verb|\mat{A}^\adjoint|     & $\mat{A}^\adjoint$ \\
    inverse              & \verb|\mat{A}^\inv|         & $\mat{A}^\inv$     \\
    \midrule
    determinant          & \verb|\det\mat{A}|          & $\det\mat{A}$      \\
    adjugate matrix      & \verb|\adj\mat{A}|          & $\adj\mat{A}$      \\
    rank                 & \verb|\rank\mat{A}|         & $\rank\mat{A}$     \\
    trace                & \verb|\trace\mat{A}|        & $\trace\mat{A}$    \\
    diagonal matrix      & \verb|\diag(a_1,\dots,a_n)|
                         & $\diag(a_1,\dots,a_n)$                           \\
    \midrule
    matrix               & \verb|\Mat{a & b \\ c & d}|
                        & $\Mat{a & b \\ c & d}$                            \\
    determinant          & \verb|\Det{a & b \\ c & d}|
                        & $\Det{a & b \\ c & d}$                            \\
    \bottomrule
  \end{tabular}
\end{center}
% \end{noindent}
\begin{gather}
  \mat{A} = (a_{ij}), \\
  \mat{I}\mat{A} = \mat{A}\mat{I} = \mat{A}, \\
  (\mat{A}\mat{B})^\transp = \mat{B}^\transp\mat{A}^\transp,\\
  \mat{A}^\adjoint = \cc{\mat{A}}^\transp, \\
  \mat{A}^\inv = \frac{1}{\det\mat{A}}\adj\mat{A}, \\
  \trace\mat{A} = \sum_{i=1}^n a_{ii}, \\
  \diag(a_1,\dots,a_n) = \Mat{a_1 & & 0 \\ & \ddots & \\ 0 & & a_n},
\end{gather}
\begin{alignat}{2}
  \mat{A}^\transp  & = \mat{A}            &  &
  \quad\lequiv\quad \mat{A}\text{ is symmetric},  \\
  \mat{A}^\adjoint & = \mat{A}            &  &
  \quad\lequiv\quad \mat{A}\text{ is hermitian},  \\
  \mat{A}^\inv     & = \mat{A}^{\transp}  &  &
  \quad\lequiv\quad \mat{A}\text{ is orthogonal}, \\
  \mat{A}^\inv     & = \mat{A}^{\adjoint} &  &
  \quad\lequiv\quad \mat{A}\text{ is unitary}.
\end{alignat}


\subsection{Vectors}
Following the ISO \num{80000}, vectors should be printed in bold italic math
font. Alternatively, using non-bold symbols with arrows is allowed.
Indices are not printed in bold. The \verb|\Vec{}| macro is implemented
similar to the previously mentioned \verb|\Mat{}| macro for matrices.
% \begin{noindent}
\begin{center}
  \begin{tabular}{llc}
    \toprule
    \textbf{Description}  & \textbf{Macro}           & \textbf{Output}     \\
    \midrule
    vector                & \verb|\vec{a}|           & $\vec{a}$           \\
    vector with arrow     & \verb|\varvec{a}|        & $\varvec{a}$        \\
    vector between points & \verb|\overrightarrow{AB}|
                          & $\overrightarrow{AB}$                          \\
    \midrule
    dot product           & \verb|\vec{a}\cdot\vec{b}|
                          & $\vec{a}\cdot\vec{b}$                          \\
    cross product         & \verb|\vec{a}\times\vec{b}|
                          & $\vec{a}\times\vec{b}$                         \\
    \midrule
    Kronecker symbol      & \verb|\kronecker{ij}|    & $\kronecker{ij}$    \\
    Levi-Civita symbol    & \verb|\leviciv{ijk}|     & $\leviciv{ijk}$     \\
    \midrule
    column vector         & \verb|\Vec{1 \\ 2 \\ 3}| & $\Vec{1 \\ 2 \\ 3}$ \\
    row vector            & \verb|\Vec{1 & 2 & 3}|   & $\Vec{1 & 2 & 3}$   \\
    \bottomrule
  \end{tabular}
\end{center}
% \end{noindent}
\begin{gather}
  \vec{v} = \Vec{v_1\\v_2\\\vdots\\v_n} = \sum_{i=1}^n v_i\,\vec{e}_i, \\
  \cos\varphi = \frac{\vec{a}\cdot\vec{b}}{\abs{\vec{a}}\abs{\vec{b}}}, \\
  \vec{a}\cdot\vec{b} = \sum_{i=1}^n a_i b_i = \kronecker{ij} a_i b_j, \\
  \vec{a}\times\vec{b} = \Vec{a_2b_3-a_3b_2 \\ a_3b_1-a_1b_3 \\ a_1b_2-a_2b_1}
  = \leviciv{ijk}\vec{e}_i a_j b_k , \\
  \vec{\dot{r}}(t) = \vec{v}(t) = \dd{\vec{r}(t)}{t}, \\
  \vec{\ddot{r}}(t) = \vec{a}(t) = \dd[2]{\vec{r}(t)}{t},
\end{gather}
\begin{gather}
  \kronecker{ij} = \begin{dcases*}
    1 & for $i=j$     \\
    0 & for $i\neq j$
  \end{dcases*}, \\
  \leviciv{ijk} = \begin{dcases*}
    1  & for $i,j,k$ cyclic       \\
    -1 & for $i,j,k$ anti-cyclic  \\
    0  & else (two equal indices)
  \end{dcases*}.
\end{gather}


\subsection{Vector spaces, norms, and inner products}
The macros for the inner product and the norm are implemented using
\texttt{mathtools} paired delimiters. They can be scaled with an optional
argument or automatically adapt to the content using the starred version.
% \begin{noindent}
\begin{center}
  \begin{tabular}{lll}
    \toprule
    \textbf{Description} & \textbf{Macro}               & \textbf{Output} \\
    \midrule
    kernel               & \verb|\ker(f)|               & $\ker(f)$       \\
    image                & \verb|\im(f)|                & $\im(f)$        \\
    dimension            & \verb|\dim(V)|               & $\dim(V)$       \\
    span                 & \verb|\vspan(v_1,\dots,v_k)|
                         & $\vspan(v_1,\dots,v_k)$                        \\
    \midrule
    inner product        & \verb|\innerp{a}{b}|         & $\innerp{a}{b}$ \\
    norm                 & \verb|\norm{x}|              & $\norm{x}$      \\
    \bottomrule
  \end{tabular}
\end{center}
% \end{noindent}
\begin{gather}
  f : V \to W, \\
  \ker(f) \coloneq \set{v \in V \given \function{f}{v} = 0 \in W}, \\
  \im(f) \coloneq \set{\function{f}{v} \in W \given v \in V},  \\
  \dim(V) = \dim\ker(f) + \dim\im(f), \\
  \dim(V_1+V_2) = \dim(V_1) + \dim(V_2) - \dim(V_1\cap V_2), \\
  \vspan(v_1,\dots,v_k)\coloneq\set*{v\in V \given
    v=\sum_{i=1}^k\lambda_i v_i,\;\lambda_i\in K},
\end{gather}
\begin{gather}
  \innerp{\cdot}{\cdot}: V \times V \to \Cnums \\
  \innerp{x+y}{z} = \innerp{x}{z} + \innerp{y}{z}, \\
  \innerp{\lambda x}{y} = \cc{\lambda}\innerp{x}{y}, \\
  \innerp{x}{y+z} = \innerp{x}{y} + \innerp{x}{z}, \\
  \innerp{x}{\lambda y} = \lambda\innerp{x}{y}, \\
  \innerp{x}{y} = \cc{\innerp{y}{x}}, \\
  \innerp{x}{x} \geq 0, \\
  \innerp{\cdot}{\cdot}: C^0(\cinterval{a,b},\Rnums)
  \times C^0(\cinterval{a,b},\Rnums) \to \Rnums, \\
  \innerp{f}{g} = \int_a^b \function{f}{t}\function{g}{t}\diff t, \\
  \innerp{\cdot}{\cdot}: L^2 \times L^2 \to \Cnums, \\
  \innerp{f}{g} = \int \cc{\function{f}{x}}\function{g}{x}\diff x, \\
  \innerp{\cdot}{\cdot}: \Cnums^{m\times n} \times \Cnums^{m\times n}
  \to \Cnums, \\
  \innerp{\mat{A}}{\mat{B}} = \trace(\mat{A}^\adjoint\mat{B}) =
  \sum_{i=1}^m\sum_{j=1}^n \cc{a}_{ij} b_{ij}.
\end{gather}
\begin{gather}
  \norm{\cdot} : V\to \Rnums_0^+, \\
  \norm{\lambda\cdot v} = \abs{\lambda}\cdot\norm{v}\quad\text{for }
  \lambda\in K \text{ and } v\in V, \\
  \norm{x+y} \leq \norm{x} + \norm{y}, \\
  \norm{v}\coloneq\sqrt{\innerp{v}{v}}.
\end{gather}


\section{Vector analysis}
This package provides the \verb|\grad|, \verb|\div|, and \verb|\rot|
operators. The capitalized macros are shorthand notations using the nabla
symbol. In addition, the Laplace operator and the D'Alembert operator are
implemented. The original div symbol $\division$ is saved as \verb|\division|.

% \begin{noindent}
\begin{center}
  \begin{tabular}{llc}
    \toprule
    \textbf{Description} & \textbf{Macro}           & \textbf{Output}     \\
    \midrule
    gradient             & \verb|\grad\phi|         & $\grad\phi$         \\
    divergence           & \verb|\div\vec{v}|       & $\div\vec{v}$       \\
    rotation             & \verb|\rot\vec{v}|       & $\rot\vec{v}$       \\
    \midrule
    nabla gradient       & \verb|\Grad\phi|         & $\Grad\phi$         \\
    nabla divergence     & \verb|\Div\vec{v}|       & $\Div\vec{v}$       \\
    nabla rotation       & \verb|\Rot\vec{v}|       & $\Rot\vec{v}$       \\
    \midrule
    Laplace operator     & \verb|\laplace\vec{v}|   & $\laplace\vec{v}$   \\
    D'Alembert operator  & \verb|\dalembert\vec{v}| & $\dalembert\vec{v}$ \\
    \bottomrule
  \end{tabular}
\end{center}
% \end{noindent}

\begin{gather}
  \nabla = \sum_{i=1}^n \vec{e}_i\pdd{}{x_i}, \\
  \laplace = \nabla^2 = \sum_{i=1}^n \pdd[2]{}{x_i}, \\
  \dalembert = \laplace - \frac{1}{c^2}\pdd[2]{}{t}, \\
  \Rot(\Rot F) = \Grad(\Div F) - \laplace F, \\
  \dalembert A^\alpha = \frac{4\pi}{c} j^\alpha, \\
  m\dd[2]{\vec{r}}{t} = -m\Grad\function{\phi}{\vec{r}}.
\end{gather}


\section{Statistics}
The probability and expectation value macros are implemented using
\texttt{mathtools} paired delimiters. They can be scaled with an optional
argument or automatically adapt to the content using the starred version.
% \begin{noindent}
\begin{center}
  \begin{tabular}{llc}
    \toprule
    \textbf{Description} & \textbf{Macro}            & \textbf{Output}      \\
    \midrule
    probability          & \verb|\proba{A}|          & $\proba{A}$          \\
    " with condition     & \verb|\proba{A \given B}| & $\proba{A \given B}$ \\
    \midrule
    expectation value    & \verb|\expect{A}|         & $\expect{A}$         \\
                         & \verb|\varexpect{A}|      & $\varexpect{A}$      \\
    variance             & \verb|\var(x)|            & $\var(x)$            \\
    covariance           & \verb|\cov(x,y)|          & $\cov(x,y)$          \\
    \midrule
    binomial coefficient & \verb|\binom{n}{k}|       & $\binom{n}{k}$       \\
    \bottomrule
  \end{tabular}
\end{center}
% \end{noindent}
\begin{gather}
  \proba{A\given B} = \frac{\proba{B\given A}\proba{A}}{\proba{B}}, \\
  \proba{A\given B} = \frac{\proba{A\cap B}}{\proba{B}}, \\
  \proba*{\bigcup_{i=1}^\infty A_i} = \sum_{i=1}^\infty\proba{A_i}, \\
  \expect{X} = \sum_{i=1}^\infty x_i\cdot\proba{X=x_i}, \\
  \expect{X} = \int_{\Rnums}\function{f}{x}\diff\function{\lambda}{x}, \\
  \var(X) \coloneq \expect*{(X-\expect{X})^2} = \expect{X^2}-\expect*{X}^2, \\
  \cov(X, Y) \coloneq \expect*{(X-\expect{X})(Y-\expect{Y})}, \\
  \cov(X, Y) = \cov(Y, X), \\
  \cov(X, X) = \var(X), \\
  \var(X) = \frac{1}{n}\sum_{i=1}^n(x_i - \bar{x})^2,\\
  \cov(X,Y) = \frac{1}{n}\sum_{i=1}^n(x_i - \bar{x})(y_i - \bar{y}), \\
  \binom{n}{k} = \frac{n!}{k!(n-k)!}.
\end{gather}


\section{Special functions}
In this section, we list special functions that are available as \LaTeX\
math operators. Missing functions were defined by this package to provide
a consistent set of macros.
\subsection{Trigonometric and hyperbolic functions}
% \begin{noindent}
\begin{center}
  \begin{tabular}{lll|lll}
    \toprule
    \textbf{Function}    & \textbf{Macro}    & \textbf{Output} &
    \textbf{Function}    & \textbf{Macro}    & \textbf{Output}   \\
    \midrule
    sine                 & \verb|\sin(x)|    & $\sin(x)$       &
    arcus sine           & \verb|\arcsin(x)| & $\arcsin(x)$      \\
    cosine               & \verb|\cos(x)|    & $\cos(x)$       &
    arcus cosine         & \verb|\arccos(x)| & $\arccos(x)$      \\
    tangent              & \verb|\tan(x)|    & $\tan(x)$       &
    arcus tangent        & \verb|\arctan(x)| & $\arctan(x)$      \\
    cotangent            & \verb|\cot(x)|    & $\cot(x)$       &
    arcus cotangent      & \verb|\arccot(x)| & $\arccot(x)$      \\
    secant               & \verb|\sec(x)|    & $\sec(x)$       &
    arcus secant         & \verb|\arcsec(x)| & $\arcsec(x)$      \\
    cosecant             & \verb|\csc(x)|    & $\csc(x)$       &
    arcus cosecant       & \verb|\arccsc(x)| & $\arccsc(x)$      \\
    \midrule
    hyperbolic sine      & \verb|\sinh(x)|   & $\sinh(x)$      &
    area hyperbolic sine & \verb|\arsinh(x)| & $\arsinh(x)$      \\
    hyp. cosine          & \verb|\cosh(x)|   & $\cosh(x)$      &
    area hyp. cosine     & \verb|\arcosh(x)| & $\arcosh(x)$      \\
    hyp. tangent         & \verb|\tanh(x)|   & $\tanh(x)$      &
    area hyp. tangent    & \verb|\artanh(x)| & $\artanh(x)$      \\
    hyp. cotangent       & \verb|\coth(x)|   & $\coth(x)$      &
    area hyp. cotangent  & \verb|\arcoth(x)| & $\arcoth(x)$      \\
    hyp. secant          & \verb|\sech(x)|   & $\sech(x)$      &
    area hyp. secant     & \verb|\arsech(x)| & $\arsech(x)$      \\
    hyp. cosecant        & \verb|\csch(x)|   & $\csch(x)$      &
    area hyp. cosecant   & \verb|\arcsch(x)| & $\arcsch(x)$      \\
    \midrule
    sinus cardinalis     & \verb|\sinc(x)|   & $\sinc(x)$      &
    \multicolumn{3}{c}{$\sinc(x) = \frac{\sin(x)}{x}$}           \\
    \bottomrule
  \end{tabular}
\end{center}
% \end{noindent}

\subsection{Logarithms}
% \begin{noindent}
\begin{center}
  \begin{tabular}{llll}
    \toprule
    \textbf{Function} & \textbf{Macro}   & \textbf{Output}
                      & \textbf{Note}                       \\
    \midrule
    logarithm         & \verb|\log_a(x)| & $\log_a(x)$
                      & $\log_a(x)=y \;\lequiv\; a^y=x$     \\
    natural logarithm & \verb|\ln(x)|    & $\ln(x)$
                      & $\ln(x) = \log_\e(x)$               \\
    decadic logarithm & \verb|\lg(x)|    & $\lg(x)$
                      & $\lg(x) = \log_{10}(x)$             \\
    binary logarithm  & \verb|\lb(x)|    & $\lb(x)$
                      & $\lb(x) = \log_2(x)$                \\
    \bottomrule
  \end{tabular}
\end{center}
% \end{noindent}


\subsection{Statistics}
% \begin{noindent}
\begin{center}
  \begin{tabular}{llll}
    \toprule
    \textbf{Function}            & \textbf{Macro}  & \textbf{Output} \\
    \midrule
    error function               & \verb|\erf(x)|  & $\erf(x)$       \\
    complementary error function & \verb|\erfc(x)| & $\erfc(x)$      \\
    \bottomrule
  \end{tabular}
\end{center}
% \end{noindent}
\begin{gather}
  \erf(x) = \frac{2}{\sqrt{\pi}}\int_0^x \e^{-t^2}\diff t, \\
  \erfc(x) = 1 - \erf(x).
\end{gather}


\subsection{Dirac delta distribution and Heaviside function}
The dirac delta distribution and the Heaviside function are implemented using
\texttt{mathtools} paired delimiters. They can be scaled with an optional
argument or automatically adapt to the content using the starred version.
% \begin{noindent}
\begin{center}
  \begin{tabular}{lll}
    \toprule
    \textbf{Function}        & \textbf{Macro}
                             & \textbf{Output}          \\
    \midrule
    dirac delta distribution & \verb|\dirac{x-x_0}|
                             & $\dirac{x-x_0}$          \\
    heaviside function       & \verb|\heaviside{x-x_0}|
                             & $\heaviside{x-x_0}$      \\
    \bottomrule
  \end{tabular}
\end{center}
% \end{noindent}
\begin{gather}
  \dirac{f} = \int_\Omega \dirac{x}f(x)\diff x = f(0), \\
  \int_\Omega \dirac*{x - \frac{1}{2}}f(x, y)\diff x
  = \function*{f}{\frac{1}{2}, y}, \\
  \heaviside{x} = \begin{cases}
    0 & x < 0    \\
    1 & x \geq 0
  \end{cases}, \\
  \dd{}{x}\heaviside{x} = \dirac{x}.
\end{gather}


\section{Physics and Engineering notation}
\subsection{Typesetting quantities}
The rules for typesetting quantities is given in the ISO \num{80000}-1.
All these rules are implemented in the package \texttt{siunitx}, which is
loaded by the \texttt{tummath} package. For detailed information it is highly
recommendet to take a look at the \texttt{siunitx} documentation on ctan
\url{https://ctan.org/pkg/siunitx}. A brief overview over the most important
rules and macros is summarized here.

\begin{itemize}
  \item Both, value and unit of a quantity must be printed in upright font and
    between value and unit there must be a space. E.g.: $l=\SI{1.2}{\meter}$.
  \item For better readability numbers can be printed in groups of three
    digits.\\E.g.: $x=\num{12345.6789}$.
  \item Units may use SI-prefixes without any space between prefix and unit.\\
    E.g.: $l=\SI{1.2}{\nano\meter}=\SI{1.2e-9}{\meter}$
  \item Between combined units, there should be a small space to avoid
    confusion with prefixes.\\E.g.: $M=\SI{120}{\newton\meter}$.
  \item In text or inline math mode, fractional units should use a division
    symbol or reciprocal notation.\\E.g.: $\unitof{V}=\si{\volt} =
      \si{\kg\m\squared\per\s\cubed\ampere} =
      \si[per-mode=reciprocal]{\kg\m\squared\per\s\cubed\ampere}$.
  \item In display math mode, fractional units may also be printed as
    fractions.
    \begin{equation}
      \unitof{\epsilon_0} = \si{\A\s\per\V\m} =
      \si{\coulomb\squared\per\N\m\squared}
    \end{equation}
  \item When using complex quantities, the complex number should be enclosed
    in parentheses.\\E.g.: $\complexqty{7+3j}{\mohm}$
  \item The standard uncerteinty of a quantity can be given in parentheses
    (provided the assumed uncertainty is given by a normal distribution).
    Printing uncertainties with the $\pm$-notation is mathematically wrong
    and should be avoided whereever possible.\\
    E.g.: $l=\SI{23.4782(32)}{\meter}$
\end{itemize}

The \texttt{tummath} package configures the default behaviour of
\texttt{siunitx} to follow these rules as close as possible.
The behaviour of all \texttt{siunitx} macros can be tweaked to specific needs
through optional arguments. With version 3.0.0 (Mai 2021) \texttt{siunitx}
changed its main document macros. Take a look at the documentation for further
details.
% \begin{noindent}
\begin{center}
  \begin{tabular}{lll}
    \toprule
    \textbf{Description} & \textbf{Macro}           & \textbf{Output}     \\
    \midrule
    number               & \verb|\num{5.34123e-9}|  & $\num{5.34123e-9}$  \\
    complex number       & \verb|\num{4.3+2.9i}|
                         & $\complexnum{4.3+2.9i}$                        \\
    " \textit{(v3)}      & \verb|\complexnum{1+i2}| & $\complexnum{1+i2}$ \\
    unit                 & \verb|\si{\nano\meter}|  & $\si{\nano\meter}$  \\
    " \textit{(v3)}      & \verb|\unit{\km\per\s}|  & $\si{\km\per\s}$    \\
    quantity             & \verb|\SI{1.23e4}{\volt\per\meter}|
                         & $\SI{1.23e4}{\volt\per\meter}$                 \\
    " \textit{(v3)}      & \verb|\qty{4.5678}{\N\per\m^2}|
                         & $\SI{4.5678}{\N\per\m^2}$                      \\
    angles decimal       & \verb|\ang{22.5}|        & $\ang{22.5}$        \\
    " deg,min,sec        & \verb|\ang{12;30;45}|    & $\ang{12;30;45}$    \\
    \bottomrule
  \end{tabular}
\end{center}
% \end{noindent}


\subsection{Unit of and angle of operators}
The unit of and angle of operators are implemented using \texttt{mathtools}
paired delimiters. They can be scaled with an optional argument or
automatically adapt to the content using the starred version.
% \begin{noindent}
\begin{center}
  \begin{tabular}{lll}
    \toprule
    \textbf{Description} & \textbf{Macro}
                         & \textbf{Output}                   \\
    \midrule
    unit of              & \verb|\unitof{x}|
                         & $\unitof{x}$                      \\
    angle of             & \verb|\angleof{\vec{a}}{\vec{b}}|
                         & $\angleof{\vec{a}}{\vec{b}}$      \\
    \bottomrule
  \end{tabular}
\end{center}
% \end{noindent}
\begin{gather}
  \unitof{P} = \si{\milli\watt}, \\
  \unitof{v} = \si{\meter\per\second}, \\
  \unitof{E}_\mathrm{SI} = \si{\volt\per\meter}, \\
  \unitof{E}_\mathrm{CGS} = \si{\statV\per\centi\meter}, \\
  \theta = \angleof*{\overline{AB}}{\overline{AC}}, \\
  \phi = \angleof*{\vec{a}}{\vec{b}}
  = \arccos\frac{\innerp*{\vec{a}}{\vec{b}}}{\norm*{\vec{a}}\norm*{\vec{b}}}.
\end{gather}


\subsection{Functionals}
The functionals macro is implemented using \texttt{mathtools} paired
delimiters. It can be scaled with an optional argument or automatically adapt
to the content using the starred version.
% \begin{noindent}
\begin{center}
  \begin{tabular}{llc}
    \toprule
    \textbf{Description} & \textbf{Macro}           & \textbf{Output}     \\
    \midrule
    functional           & \verb|\functional{S}{y}| & $\functional{S}{y}$ \\
    \bottomrule
  \end{tabular}
\end{center}
% \end{noindent}
\begin{gather}
  \functional[\big]{S}{\function{q}{t}} = \int_{t_1}^{t_2}
  \function{\mathcal{L}}{q,\dot{q},t} \diff t \\
  \delta \functional{S}{q} = 0 \quad\lequiv\quad
  \dd{}{t}\pdd{\mathcal{L}}{\dot{q}_i} = \pdd{\mathcal{L}}{q_i}
\end{gather}


\subsection{Bra-Ket notation}
All the bra-ket macros are implemented using \texttt{mathtools} paired
delimiters. They can be scaled with an optional argument or automatically
adapt to the content using the starred version. Inside the \verb|\braket{}|
macro, the pipe symbol is redefined to print a vertical bar that automatically
adapts to the delimiter size. The pipe symbol can be used arbitrarily often
inside the braket macro.
% \begin{noindent}
\begin{center}
  \begin{tabular}{lll}
    \toprule
    \textbf{Description} & \textbf{Macro}    & \textbf{Output} \\
    \midrule
    operator             & \verb|\op{H}|     & $\op{H}$        \\
    \midrule
    bra                  & \verb|\bra{\psi}| & $\bra{\psi}$    \\
    ket                  & \verb|\ket{\psi}| & $\ket{\psi}$    \\
    \midrule
    bra-ket              & \verb"\braket{\psi | A | \psi}"
                         & $\braket{\psi | A | \psi}$          \\[.5ex]
    (inner product)      & \verb"\braket{\psi | \phi}"
                         & $\braket{\psi|\phi}$                \\[.5ex]
    (expectation value)  & \verb"\braket{\op{A}}"
                         & $\braket{\op{A}}$                   \\
    \midrule
    ket-bra              & \verb|\ketbra{\psi}{\phi}|
                         & $\ketbra{\psi}{\phi}$               \\
    \bottomrule
  \end{tabular}
\end{center}
% \end{noindent}
\begin{gather}
  \ket{\alpha}+\ket{\beta} = \ket{\alpha+\beta},\\
  \i\hbar\ket*{\dot{\psi}(t)} = \op{H}\ket{\psi(t)}, \\
  \ket*{\dd{\psi(t)}{t}} = -\frac{\i}{\hbar}\op{H}\ket{\psi(t)}, \\
  \braket{\alpha | \beta} = \cc{\braket{\beta | \alpha}}, \\
  \op{\rho} = \ketbra{\psi}{\psi}, \\
  \op{\rho} = \sum_i p_i \ketbra{\psi_i}{\psi_i}, \\
  \braket*{\psi | \op{A} | \phi}, \\
  \braket*{\psi | \commutator{\op{A}}{\op{B}} | \psi}.
\end{gather}


\subsection{Commutators and Poisson Brackets}
The poisson bracket and the commutators are implemented using
\texttt{mathtools} paired delimiters. They can be scaled with an optional
argument or automatically adapt to the content using the starred version.
% \begin{noindent}
\begin{center}
  \begin{tabular}{lll}
    \toprule
    \textbf{Description} & \textbf{Macro}
                         & \textbf{Output}              \\
    \midrule
    poisson bracket      & \verb|\poissonbracket{f}{g}|
                         & $\poissonbracket{f}{g}$      \\
    commutator           & \verb|\commutator{A}{B}|
                         & $\commutator{A}{B}$          \\
    anti-commutator      & \verb|\anticommutator{A}{B}|
                         & $\anticommutator{A}{B}$      \\
    \bottomrule
  \end{tabular}
\end{center}
% \end{noindent}
\begin{gather}
  \poissonbracket{f}{g} = \sum_{j=1}^S\left(
    \pdd{f}{q_j}\pdd{g}{p_j} - \pdd{f}{p_j}\pdd{g}{q_j} \right), \\
  \dd{f}{t}=\poissonbracket{f}{H}+\pdd{f}{t}, \\
  \dot{q}_j = \poissonbracket{q_j}{H}, \\
  \dot{p}_j = \poissonbracket{p_j}{H}, \\
  \pdd{}{t}\poissonbracket{f}{g} = \poissonbracket*{\pdd{f}{t}}{g} +
  \poissonbracket*{f}{\pdd{g}{t}},
\end{gather}
\begin{gather}
  \commutator*{\op{A}}{\op{B}} = \op{A}\op{B} - \op{B}\op{A}, \\
  \dd{\op{A}}{t} = \commutator*{\op{A}}{\op{H}} + \pdd{\op{A}}{t}, \\
  \dot{\op{q}}_i = \frac{1}{\i\hbar}\commutator*{\op{q}_i}{\op{H}}, \\
  \dot{\op{p}}_i = \frac{1}{\i\hbar}\commutator*{\op{p}_i}{\op{H}}, \\
  \anticommutator*{\op{A}}{\op{B}} = \op{A}\op{B} + \op{B}\op{A}, \\
\end{gather}


\subsection{Tensors}
To typeset tensors in index notation, the \texttt{tummath} package loads the
\texttt{tensor} package. For detailed information take a look at the package
documentation on ctan \url{https://ctan.org/pkg/tensor}.
% \begin{noindent}
\begin{center}
  \begin{tabular}{lll}
    \toprule
    \textbf{Description}     & \textbf{Macro}       & \textbf{Output} \\
    \midrule
    tensor & \verb|\tensor{T}{^a_b^c_d}| & $\tensor{T}{^a_b^c_d}$ \\
    \bottomrule
  \end{tabular}
\end{center}
% \end{noindent}
\begin{gather}
  a^\prime_\alpha = \nu_{\alpha\beta}\tensor{\Lambda}{^\beta_\gamma}
  \nu^{\gamma\delta}a_\delta = \tensor{\Lambda}{_\alpha^\delta}a_\delta, \\
  T^{\prime\,\alpha_1\alpha_2\dots\alpha_n} =
  \tensor{\Lambda}{^{\alpha_1}_{\beta_1}}
  \tensor{\Lambda}{^{\alpha_2}_{\beta_2}}\dots
  \tensor{\Lambda}{^{\alpha_n}_{\beta_n}}T^{\beta_1\beta_2\dots\beta_2}, \\
  \tensor{\Gamma}{^\kappa_\mu_\lambda} = \frac{g^{\kappa\nu}}{2}
  \left(\pdd{g_{\mu\nu}}{x^\lambda}+\pdd{g_{\lambda\nu}}{x^\mu}-
    \pdd{g_{\mu\lambda}}{x^\nu}\right), \\
  V_{\mu;\nu}=V_{\mu,\nu}-\tensor{\Gamma}{^\lambda_\mu_\nu}V_\lambda.
\end{gather}


\subsection{Transformations}
This package provides macros for the transform symbols. The optional argument
may be used to clarify the type of transformation if necessary.
% \begin{noindent}
\begin{center}
  \begin{tabular}{lll}
    \toprule
    \textbf{Description}     & \textbf{Macro}       & \textbf{Output} \\
    \midrule
    transform symbol         & \verb|\transform|    & $\transform$    \\
    inverse transform symbol & \verb|\itransform|   & $\itransform$   \\
    \midrule
    Fourier transform        & \verb|\transform[F]| & $\transform[F]$ \\
    Laplace transform        & \verb|\transform[L]| & $\transform[L]$ \\
    Z transform              & \verb|\transform[Z]| & $\transform[Z]$ \\
    \bottomrule
  \end{tabular}
\end{center}
% \end{noindent}
\begin{gather}
  \function{F}{\omega} = \mathcal{F}\left\lbrace\function{f}{t}\right\rbrace
  = \frac{1}{\sqrt{2\pi}}\int_{-\infty}^{\infty}
  \function{f}{t}\e^{-\i\omega t} \diff t, \\
  \function{f}{t} = \mathcal{F}^\inv\lbrace\function{F}{\omega}\rbrace
  = \frac{1}{\sqrt{2\pi}}\int_{-\infty}^{\infty}
  \function{F}{\omega}\e^{\i\omega t} \diff \omega.
\end{gather}
\begin{gather}
  \function{F}{\varOmega} = \mathrm{DTFT}\lbrace f_k \rbrace
  = \sum_{k=-\infty}^\infty f_k \e^{-\i\varOmega k}, \\
  f_k = \mathrm{DTFT}^\inv\lbrace\function{F}{\varOmega}\rbrace
  = \frac{1}{2\pi}\int_{-\pi}^{\pi}
  \function{F}{\varOmega}\e^{\i\varOmega k} \diff \varOmega, \\
  \mathrm{with}\;\varOmega = \omega\cdot T_A.
\end{gather}
\begin{gather}
  \function{F}{s} = \mathcal{L}\lbrace\function{f}{t}\rbrace
  = \int_0^\infty \function{f}{t}\e^{-st}\diff t, \\
  \function{f}{t} = \mathcal{L}^\inv\lbrace\function{F}{s}\rbrace
  = \frac{1}{2\pi\i}\int_{\gamma-\i\infty}^{\gamma+\i\infty}
  \function{F}{s}\e^{st} \diff s, \\
  \mathrm{with}\; s\in\Cnums,\; \gamma > \Re{s_0}.
\end{gather}
\begin{gather}
  \function{F}{z} = \mathcal{Z}\lbrace f_k \rbrace
  = \sum_{k=0}^\infty f_k z^{-k}, \\
  f_k = \mathcal{Z}^\inv\lbrace\function{F}{z}\rbrace
  = \frac{1}{2\pi\i}\oint_C \function{F}{z} z^{k-1} \diff z, \\
  \mathrm{with}\; z = \e^{sT_A}.
\end{gather}

\begin{gather}
  f(t-a) \transform[F] \e^{-\i a\omega}F(\omega), \\
  F(\omega-a) \itransform[F] \e^{\i at}f(t), \\
  f^{(n)}(t) \transform[F] (\i\omega)^nF(\omega),
\end{gather}
\begin{gather}
  f(t-a) \transform[L] \e^{-as}F(s), \\
  F(s-a) \itransform[L] \e^{at}f(t), \\
  f^{(n)}(t) \transform[L] s^n F(s) - \sum_{k=0}^{n-1} s^{n-k-1} f^{(k)}(0),
\end{gather}
\begin{gather}
  f_{k-n} \transform[Z] z^{-n}F(z), \\
  F(az) \itransform[Z] a^{-k} f_k,
\end{gather}
\begin{gather}
  f_{k-n} \transform[\mathrm{DTFT}] \e^{-\i\varOmega n} F(\varOmega), \\
  F(\varOmega-a) \itransform[\mathrm{DTFT}] \e^{\i ak}f_k.
\end{gather}

\end{document}
