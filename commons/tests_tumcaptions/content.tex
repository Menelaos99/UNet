\blindtext

\begin{figure}
  \centering
  \includegraphics[width=.5\textwidth]{example-image-a}
  \caption{Very short caption.}
\end{figure}

\begin{figure}
  \centering
  \includegraphics[width=.5\textwidth]{example-image-a}
  \caption{This, on the other hand, is very long caption. It is so long, that
    during the time it takes to read it kingdoms may rise and fall again.}
\end{figure}

\begin{figure}
  \begin{subfigure}[t]{.4\textwidth}
    \centering
    \includegraphics[width=0.8\textwidth]{example-image-a}
    \subcaption{Subfigure left, short caption.}
    \label{fig:grid_a}
  \end{subfigure}
  \hfill
  \begin{subfigure}[t]{.5\textwidth}
    \centering
    \includegraphics[width=0.8\textwidth]{example-image-a}
    \subcaption{Subfigure right has a long long long long long long caption.}
    \label{fig:grid_b}
  \end{subfigure}
  \centering
  \caption{Schematic overview of the FDTD method. The circles and crosses
    represent the spatiotemporal discretization of the magnetic field and
    electric field. Additionally, the crosses denote the positions of the
    quantum mechanical systems.}
  \label{fig:grid}
\end{figure}

\begin{table}
  \caption{Very short caption.}
  \centering
  \begin{tabular}{lcr}
    \toprule
    Header 1 & Header 2 & Header 3 \\
    \midrule
    Data 1   & 3.5      & 4.3      \\
    Data 2   & 3.5      & 4.3      \\
    Data 3   & 3.5      & 4.3      \\
    Data 4   & 3.5      & 4.3      \\
    Data 5   & 3.5      & 4.3      \\
    \bottomrule
  \end{tabular}
\end{table}

\begin{table}
  \caption{This, on the other hand, is very long caption. It is so long, that
    during the time it takes to read it kingdoms may rise and fall again.}
  \centering
  \begin{tabular}{lcr}
    \toprule
    Header 1 & Header 2 & Header 3 \\
    \midrule
    Data 1   & 3.5      & 4.3      \\
    Data 2   & 3.5      & 4.3      \\
    Data 3   & 3.5      & 4.3      \\
    Data 4   & 3.5      & 4.3      \\
    Data 5   & 3.5      & 4.3      \\
    \bottomrule
  \end{tabular}
\end{table}

\begin{algorithm}[H]
  \centering
  \begin{algorithmic}
    \FOR{$n = 0$ \TO $n_\mathrm{max}$}

    \FOR{$m = 1$ \TO $m_\mathrm{max}$}
    \STATE{$h[m] \gets \mathrm{update\_h}(e[m], e[m-1])$}
    \STATE{$d[m] \gets \mathrm{update\_d}(e[m])$}
    \STATE{$p[m] \gets \mathrm{calc\_p}(d[m])$}
    \ENDFOR
    \STATE{sync()}

    \FOR{$m = 0$ \TO $m_\mathrm{max}-1$}
    \STATE{$e[m] \gets \mathrm{update\_e}(h[m + 1], h[m], p[m])$}
    \ENDFOR
    \STATE{sync()}
    \ENDFOR
  \end{algorithmic}
  \caption{Simulation main loop -- basic version.}
  \label{alg:loop_basic}
\end{algorithm}

\begin{algorithm}[H]
  \centering
  \begin{algorithmic}
    \FOR{$n = 0$ \TO $n_\mathrm{max}$}

    \FOR{$m = 1$ \TO $m_\mathrm{max}$}
    \STATE{$h[m] \gets \mathrm{update\_h}(e[m], e[m-1])$}
    \STATE{$d[m] \gets \mathrm{update\_d}(e[m])$}
    \STATE{$p[m] \gets \mathrm{calc\_p}(d[m])$}
    \ENDFOR
    \STATE{sync()}

    \FOR{$m = 0$ \TO $m_\mathrm{max}-1$}
    \STATE{$e[m] \gets \mathrm{update\_e}(h[m + 1], h[m], p[m])$}
    \ENDFOR
    \STATE{sync()}
    \ENDFOR
  \end{algorithmic}
  \caption{Simulation main loop -- basic version. The same algorithm but with
    a way longer, way way longer, I would even say way, way, way longer
    caption.}
  \label{alg:loop_basic2}
\end{algorithm}

Finally, the Listing~\ref{lst:overlap} is beautiful.

\begin{lstlisting}[caption={Overlap of computation and communication.},
  label={lst:overlap}]
int main(int argc, char **argv) {
    return 0;
}
\end{lstlisting}

\begin{lstlisting}[caption={Overlap of computation and communication. Here
  with a significantly longer caption. Should exceed a line length, I
  believe. Just keep on typing until it does.}, label={lst:overlap2}]
int main(int argc, char **argv) {
    return 0;
}
\end{lstlisting}
