\tableofcontents

\section{University name}

Embed the name of the university (\theUniversityName{}) in text. For example,
the \theUniversityName{} is in Munich.

\section{Department and group names}

Those can be adjusted by overwriting the default values in
\texttt{tumuser.sty}. Without modifications, the department is
\theDepartmentName{} and the group macro contains ``\theChairName{}''.

\section{Switching languages}

The templates use the babel package for language support. Supported languages
are German (ngerman in babel) and English (english in babel). If you need to
switch between languages within a document, you can use
\begin{verbatim}
\selectlanguage{X}
\end{verbatim}
where X is either ngerman or english.

It is possible to define a macro that adapts its content to the active
language. For example, the macro that prints the university name is defined
as
\begin{verbatim}
\provideName{\theUniversityName}{Technical University of Munich}{Technische
  Universität München}
\end{verbatim}
