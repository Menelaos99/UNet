\documentclass{scrartcl}

\usepackage{tumbase}
\usepackage{tumcolors}
\usepackage{tumlogo}
\usepackage{tumlang}
\usepackage{tumcaptions}
\PassOptionsToPackage{showframe}{geometry}
\usepackage[@PAPER@, @ORIENTATION@]{tumpage}

\usepackage{helvet}
\renewcommand{\familydefault}{\sfdefault}

\usepackage{booktabs}
\usepackage{array}
\newcolumntype{^}{>{\global\let\currentrowstyle\relax}}
\newcolumntype{'}{>{\currentrowstyle}}
\newcommand{\rowstyle}[1]{\gdef\currentrowstyle{#1}#1\ignorespaces}

\usepackage{etoolbox}
\usepackage{pgfmath}
\usepackage{printlen}

\newcommand{\prln}[2][mm]{\uselengthunit{#1}\rndprintlength{#2}}

\newcommand{\acvd}[2]{%
  % round value (in units pt) for comparison
  \pgfmathparse{round(#1) == round(#2)}%
  \ifnumequal{\pgfmathresult}{1}{%
    \cellcolor{TUMGreen!20}\color{TUMGreen}\bfseries OK%
  }{%
    \cellcolor{TUMExtRed!20}\color{TUMExtRed}\bfseries FAIL%
  }%
}

\newlength{\thefsize}
\newlength{\theblskip}

\makeatletter
\newcommand{\getfsize}[1]{{#1%
    \global\setlength{\thefsize}{\f@size pt}%
    \global\setlength{\theblskip}{\baselineskip}%
  }%
}
\makeatother

\newcommand{\normaltest}{%
  \pgfmathparse{round(\thefsize) == round(\refbasefont)}%
  \ifnumequal{\pgfmathresult}{1}{%
    \cellcolor{TUMGreen!20}}{\cellcolor{TUMExtRed!20}}%
}


% load reference values for layout dimensions and font sizes
\newlength{\reftopmrgn}
\newlength{\refbottommrgn}
\newlength{\refinnermrgn}
\newlength{\refoutermrgn}
\newlength{\refheadheight}
\newlength{\refheadsep}
\newlength{\reffootskip}
\newlength{\refmarginparwidth}
\newlength{\refmarginparsep}
\newlength{\refcolumnsep}
\newlength{\refpaperheight}
\newlength{\refpaperwidth}
\newlength{\refonefont}
\newlength{\reftwofont}
\newlength{\refthreefont}
\newlength{\reffourfont}
\newlength{\refbasefont}


% .:: Define lengths according to paper size and orientation
% ----------------------------------------------------------------------------
% Values taken from TUM CD Handbook and extracted from PowerPoint Templates
\makeatletter
\ifdefstring{\@tum@papername@}{a0paper}{
  \deflength{\reftopmrgn}{138mm}
  \deflength{\refbottommrgn}{46mm}
  \deflength{\refinnermrgn}{46mm}
  \deflength{\refoutermrgn}{46mm}
  \deflength{\refheadheight}{46mm}
  \deflength{\refheadsep}{46mm}
  \deflength{\reffootskip}{0mm}
  \deflength{\refmarginparwidth}{0mm}
  \deflength{\refmarginparsep}{0mm}
  \deflength{\refcolumnsep}{18mm}
  \ifbool{@tum@landscape@}{
    \deflength{\refpaperheight}{841mm}
    \deflength{\refpaperwidth}{1189mm}
  }{
    \deflength{\refpaperheight}{1189mm}
    \deflength{\refpaperwidth}{841mm}
  }
  \deflength{\refonefont}{94pt}
  \deflength{\reftwofont}{67pt}
  \deflength{\refthreefont}{44pt}
  \deflength{\reffourfont}{30pt}
  \deflength{\refbasefont}{25pt}
}{}
\ifdefstring{\@tum@papername@}{a1paper}{
  \deflength{\reftopmrgn}{105mm}
  \deflength{\refbottommrgn}{35mm}
  \deflength{\refinnermrgn}{35mm}
  \deflength{\refoutermrgn}{35mm}
  \deflength{\refheadheight}{35mm}
  \deflength{\refheadsep}{35mm}
  \deflength{\reffootskip}{0mm}
  \deflength{\refmarginparwidth}{0mm}
  \deflength{\refmarginparsep}{0mm}
  \deflength{\refcolumnsep}{14mm}
  \ifbool{@tum@landscape@}{
    \deflength{\refpaperheight}{594mm}
    \deflength{\refpaperwidth}{841mm}
  }{
    \deflength{\refpaperheight}{841mm}
    \deflength{\refpaperwidth}{594mm}
  }
  \deflength{\refonefont}{67pt}
  \deflength{\reftwofont}{49pt}
  \deflength{\refthreefont}{39pt}
  \deflength{\reffourfont}{26pt}
  \deflength{\refbasefont}{22pt}
}{}
\ifdefstring{\@tum@papername@}{a2paper}{
  \deflength{\reftopmrgn}{81mm}
  \deflength{\refbottommrgn}{27mm}
  \deflength{\refinnermrgn}{27mm}
  \deflength{\refoutermrgn}{27mm}
  \deflength{\refheadheight}{27mm}
  \deflength{\refheadsep}{27mm}
  \deflength{\reffootskip}{0mm}
  \deflength{\refmarginparwidth}{0mm}
  \deflength{\refmarginparsep}{0mm}
  \deflength{\refcolumnsep}{10mm}
  \ifbool{@tum@landscape@}{
    \deflength{\refpaperheight}{420mm}
    \deflength{\refpaperwidth}{594mm}
  }{
    \deflength{\refpaperheight}{594mm}
    \deflength{\refpaperwidth}{420mm}
  }
  \deflength{\refonefont}{48pt}
  \deflength{\reftwofont}{35pt}
  \deflength{\refthreefont}{28pt}
  \deflength{\reffourfont}{20pt}
  \deflength{\refbasefont}{18pt}
}{}
\ifdefstring{\@tum@papername@}{a3paper}{
  \deflength{\reftopmrgn}{54mm}
  \deflength{\refbottommrgn}{18mm}
  \deflength{\refinnermrgn}{18mm}
  \deflength{\refoutermrgn}{18mm}
  \deflength{\refheadheight}{18mm}
  \deflength{\refheadsep}{18mm}
  \deflength{\reffootskip}{0mm}
  \deflength{\refmarginparwidth}{0mm}
  \deflength{\refmarginparsep}{0mm}
  \deflength{\refcolumnsep}{7mm}
  \ifbool{@tum@landscape@}{
    \deflength{\refpaperheight}{297mm}
    \deflength{\refpaperwidth}{420mm}
  }{
    \deflength{\refpaperheight}{420mm}
    \deflength{\refpaperwidth}{297mm}
  }
  \deflength{\refonefont}{34pt}
  \deflength{\reftwofont}{25pt}
  \deflength{\refthreefont}{20pt}
  \deflength{\reffourfont}{16pt}
  \deflength{\refbasefont}{16pt}
}{}
\ifdefstring{\@tum@papername@}{a4paper}{
  \deflength{\reftopmrgn}{30mm}
  \deflength{\refbottommrgn}{10mm}
  \deflength{\refinnermrgn}{10mm}
  \deflength{\refoutermrgn}{10mm}
  \deflength{\refheadheight}{10mm}
  \deflength{\refheadsep}{10mm}
  \deflength{\reffootskip}{0mm}
  \deflength{\refmarginparwidth}{0mm}
  \deflength{\refmarginparsep}{0mm}
  \deflength{\refcolumnsep}{5mm}
  \ifbool{@tum@landscape@}{
    \deflength{\refpaperheight}{210mm}
    \deflength{\refpaperwidth}{297mm}
  }{
    \deflength{\refpaperheight}{297mm}
    \deflength{\refpaperwidth}{210mm}
  }
  \deflength{\refonefont}{24pt}
  \deflength{\reftwofont}{18pt}
  \deflength{\refthreefont}{14pt}
  \deflength{\reffourfont}{11pt}
  \deflength{\refbasefont}{11pt}
}{}
\ifdefstring{\@tum@papername@}{a5paper}{
  \deflength{\reftopmrgn}{21mm}
  \deflength{\refbottommrgn}{7mm}
  \deflength{\refinnermrgn}{7mm}
  \deflength{\refoutermrgn}{7mm}
  \deflength{\refheadheight}{7mm}
  \deflength{\refheadsep}{7mm}
  \deflength{\reffootskip}{0mm}
  \deflength{\refmarginparwidth}{0mm}
  \deflength{\refmarginparsep}{0mm}
  \deflength{\refcolumnsep}{5mm}
  \ifbool{@tum@landscape@}{
    \deflength{\refpaperheight}{148mm}
    \deflength{\refpaperwidth}{210mm}
  }{
    \deflength{\refpaperheight}{210mm}
    \deflength{\refpaperwidth}{148mm}
  }
  \deflength{\refonefont}{17pt}
  \deflength{\reftwofont}{13pt}
  \deflength{\refthreefont}{11pt}
  \deflength{\reffourfont}{9pt}
  \deflength{\refbasefont}{9pt}
}{}
\makeatother


% loading the desired TUM pagestyle
\pagestyle{TUM.titlepage}

% calculating page margins from TeX lengths
% see https://en.wikibooks.org/wiki/LaTeX/Page_Layout
\newlength{\topmrgn}
\deflength{\topmrgn}{1in+\voffset+\topmargin+\headheight+\headsep}
\newlength{\bottommrgn}
\deflength{\bottommrgn}{\paperheight-\topmrgn-\textheight}
\newlength{\innermrgn}
\deflength{\innermrgn}{1in+\hoffset+\oddsidemargin}
\newlength{\outermrgn}
\deflength{\outermrgn}{\paperwidth-\innermrgn-\textwidth}


\begin{document}
\noindent
This page uses the \textbf{TUM.titlepage} layout.\\
You should see the TUM Logo and Threeliner in the headline printed
in \textbf{\color{TUMBlue}TUMBlue}.

\begin{table}[ht]
  \caption{Layout dimensions check.}
  \centering
  \begin{tabular}{^l'r'r'c}
    \toprule
    \rowstyle{\bfseries}
    length name    & real value
                   & reference value
                   & achieved                                   \\
    \midrule
    top margin     & \prln{\topmrgn}
                   & \prln{\reftopmrgn}
                   & \acvd{\topmrgn}{\reftopmrgn}               \\
    bottom margin  & \prln{\bottommrgn}
                   & \prln{\refbottommrgn}
                   & \acvd{\bottommrgn}{\refbottommrgn}         \\
    inner margin   & \prln{\innermrgn}
                   & \prln{\refinnermrgn}
                   & \acvd{\innermrgn}{\refinnermrgn}           \\
    outer margin   & \prln{\outermrgn}
                   & \prln{\refoutermrgn}
                   & \acvd{\outermrgn}{\refoutermrgn}           \\
    \midrule
    headheight     & \prln{\headheight}
                   & \prln{\refheadheight}
                   & \acvd{\headheight}{\refheadheight}         \\
    headsep        & \prln{\headsep}
                   & \prln{\refheadsep}
                   & \acvd{\headsep}{\refheadsep}               \\
    footskip       & \prln{\footskip}
                   & \prln{\reffootskip}
                   & \acvd{\footskip}{\reffootskip}             \\
    \midrule
    marginparwidth & \prln{\marginparwidth}
                   & \prln{\refmarginparwidth}
                   & \acvd{\marginparwidth}{\refmarginparwidth} \\
    marginparsep   & \prln{\marginparsep}
                   & \prln{\refmarginparsep}
                   & \acvd{\marginparsep}{\refmarginparsep}     \\
    columnsep      & \prln{\columnsep}
                   & \prln{\refcolumnsep}
                   & \acvd{\columnsep}{\refcolumnsep}           \\
    \midrule
    paperheight    & \prln{\paperheight}
                   & \prln{\refpaperheight}
                   & \acvd{\paperheight}{\refpaperheight}       \\
    paperwidth     & \prln{\paperwidth}
                   & \prln{\refpaperwidth}
                   & \acvd{\paperwidth}{\refpaperwidth}         \\
    textheight     & \prln{\textheight}
                   & ---
                   & ---                                        \\
    textwidth      & \prln{\textwidth}
                   & ---
                   & ---                                        \\
    \bottomrule
  \end{tabular}
\end{table}

\begin{table}[ht]
  \caption{Font sizes specified by the TUM Corporate Design.}
  \centering
  \begin{tabular}{^l'r'r'c}
    \toprule
    \rowstyle{\bfseries}
    size name & fontsize
              & reference value
              & achieved                        \\
    \midrule
    \getfsize{\usekomafont{TUM.H1}}%
    TUM H1    & \prln[pt]{\thefsize}
              & \prln[pt]{\refonefont}
              & \acvd{\thefsize}{\refonefont}   \\
    \getfsize{\usekomafont{TUM.H2}}%
    TUM H2    & \prln[pt]{\thefsize}
              & \prln[pt]{\reftwofont}
              & \acvd{\thefsize}{\reftwofont}   \\
    \getfsize{\usekomafont{TUM.H3}}%
    TUM H3    & \prln[pt]{\thefsize}
              & \prln[pt]{\refthreefont}
              & \acvd{\thefsize}{\refthreefont} \\
    \getfsize{\usekomafont{TUM.H4}}%
    TUM H4    & \prln[pt]{\thefsize}
              & \prln[pt]{\reffourfont}
              & \acvd{\thefsize}{\reffourfont}  \\
    \getfsize{\usekomafont{TUM.base}}%
    TUM Base  & \prln[pt]{\thefsize}
              & \prln[pt]{\refbasefont}
              & \acvd{\thefsize}{\refbasefont}  \\
    \bottomrule
  \end{tabular}
\end{table}

\begin{table}[ht]
  \caption{Summary of available \LaTeX\ font sizes.\\
    Normalsize should be set equal to TUM Base.}
  \centering
  \begin{tabular}{^l'l'l}
    \toprule
    \rowstyle{\bfseries}
    size name    & fontsize             & baselineskip          \\
    \midrule
    \getfsize{\Huge}%
    Huge         & \prln[pt]{\thefsize} & \prln[pt]{\theblskip} \\
    \getfsize{\huge}%
    huge         & \prln[pt]{\thefsize} & \prln[pt]{\theblskip} \\
    \getfsize{\LARGE}%
    LARGE        & \prln[pt]{\thefsize} & \prln[pt]{\theblskip} \\
    \getfsize{\Large}%
    Large        & \prln[pt]{\thefsize} & \prln[pt]{\theblskip} \\
    \getfsize{\large}%
    large        & \prln[pt]{\thefsize} & \prln[pt]{\theblskip} \\
    \midrule
    \getfsize{\normalsize}\rowstyle{\normaltest}%
    normal       & \prln[pt]{\thefsize} & \prln[pt]{\theblskip} \\
    \getfsize{\small}%
    small        & \prln[pt]{\thefsize} & \prln[pt]{\theblskip} \\
    \getfsize{\footnotesize}%
    footnotesize & \prln[pt]{\thefsize} & \prln[pt]{\theblskip} \\
    \getfsize{\scriptsize}%
    scriptsize   & \prln[pt]{\thefsize} & \prln[pt]{\theblskip} \\
    \getfsize{\tiny}%
    tiny         & \prln[pt]{\thefsize} & \prln[pt]{\theblskip} \\
    \bottomrule
  \end{tabular}
\end{table}
\end{document}
