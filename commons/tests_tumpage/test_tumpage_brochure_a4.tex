\documentclass[parskip=on, twoside]{scrartcl}

\usepackage{tumbase}
\usepackage{tumcolors}
\usepackage{tumlogo}
\usepackage[german]{tumlang}
\usepackage{tumcaptions}
\PassOptionsToPackage{showframe}{geometry}
\usepackage[papername=a4paper, headline=threeliner]{tumpage}
\usepackage{tumfonts}

\usepackage{graphicx}
\usepackage{multicol}
\usepackage{ragged2e}
\usepackage{url}


\begin{document}

% .:: Page 1
% ----------------------------------------------------------------------------
\pagestyle{TUM.titlepage}
{\usekomafont{TUM.H1}Überschrift 1 läuft über gesamte Papierbreite\par}
{\usekomafont{TUM.H2} Überschrift 2 läuft über die gesamte Papierbreite\par}
\strut\par
\includegraphics[width=\textwidth, height=0.85\textheight]{example-image}

% .:: Page 2
% ----------------------------------------------------------------------------
\pagestyle{TUM.insidepage}
{\usekomafont{TUM.H1} Mustervorlage Broschüre DIN A4\par}
\strut\par
\begin{multicols*}{2}
  {\usekomafont{TUM.H2} Die Broschüre ist ein mehrseitiges Heft (mindestens
    8 Seiten).\par}
  {\usekomafont{TUM.H3} Für ein 4-seitiges Dokument wählen Sie bitte
    die Vorlage für ein Faltblatt.\par}

  \paragraph{Schriftgröße 11pt}\strut\\
  Dies ist die Vorlage für eine Broschüre (DIN A4, Hochformat) der Technischen
  Universität München (TUM). Sie entspricht dem Corporate Design der TUM und
  ist für das Betriebssystem Windows getestet. Sie ist mit Office-Versionen ab
  2007 kompatibel.\par
  Bitte geben Sie Ihren individuellen Text an den vorgesehenen Stellen ein und
  nutzen Sie die installierten Formatvorlagen für den jeweiligen
  Textabschnitt.

  \paragraph{Kopfzeile und Absender}\strut\\
  Es stehen mehrere Varianten zur Auswahl. Diese finden Sie auf den nächsten
  Seiten als Beispiel oder unter www.tum.de/cd, welche für die verschiedenen
  Nutzergruppen vorgesehen sind. Außerdem sind die Kopfzeilenvarianten bereits
  in den Schnellbausteinen angelegt. Wählen Sie dazu das Menü "Organizer für
  Schnellbausteine" und klicken die entsprechende Kopfzeile der TUM an. Sie
  gelangen auch über das „Menüband“ unter dem Reiter „Kopfzeile“ zu diesen
  Textbausteinen.

  \paragraph{Text}\strut\\
  Grundsätzlich steht der Text linksbündig oder als linksbündiger Blocksatz.
  Aber Sie haben auch hier Varianten zur Auswahl:
  \begin{itemize}
    \item Legen Sie den Text in zwei Spalten an. Das ist die übliche
      Textanordnung in diesem Format.
    \item Legen Sie den Text in einer Spalte an. Diese Variante ist weniger
      empfehlenswert, da sie besonders bei großen Textmengen nicht
      leserfreundlich ist.
    \item Legen Sie den Text in einer Spalte mit einer Marginalspalte an. Die
      Marginalspalte kann Zusatzinformationen oder Randbemerkungen enthalten
      und ist deshalb für erklärende oder wissenschaftliche Texte gut
      geeignet.
  \end{itemize}
  Die Varianten sollten nicht oder nur selten innerhalb eines Dokuments
  wechseln. Auf den folgenden Seiten sehen Sie Beispiele der Text-Varianten.
\end{multicols*}

% .:: Page 3 and 4
% ----------------------------------------------------------------------------
\pagestyle{TUM.margincolumn}
\strut\marginpar{\RaggedRight\footnotesize Die kleine sogenannte
  Mar\-gi\-nal\-spal\-te außen liefert Zusatzinformationen oder
  Randbemerkungen zum Haupttext nebenan.}

\vspace{-\baselineskip}\vspace{-\parskip}
\begingroup
\includegraphics[width=\textwidth, height=.5\textheight]{example-image-a}
\captionof{figure}{Bildunterschrift, Autor etc.}
\endgroup

\paragraph{Bilder}\strut\\
Auch wenn Sie Bilder integrieren möchten, können Sie innerhalb des
vorgegebenen Rahmens dieser Vorlage variieren:
\begin{itemize}
  \item Das Bild ist an die Breite der Textspalte angepasst
  \item Das Bild ist seitenfüllend innerhalb des vorgesehenen Textrahmens (mit
    weißem Rand)
  \item Das Bild ist seitenfüllend bis an den Papierrand (randabfallend) --
    an mindestens drei Seiten.
\end{itemize}
\marginpar{\RaggedRight\footnotesize Der Begriff Aufläsung taucht z.B. im
  Zusammenhang mit Fernsehbildschirmen oder Digitalkameras auf. Hier
  beschreibt Auflösung die Bildqualität.}

Bilder sollten immer eine Bildunterschrift tragen, die ggf. Informationen zum
Bild und zur Autorenschaft enthält.\par

Das Bild sollte außerdem immer genügend Abstand zum Text haben. Gleiches gilt
für Grafiken.\par
\marginpar{\RaggedRight\footnotesize Nur mit einer entsprechend hohen
  Auflösung wirken Bilder in Druckerzeugnissen nicht verschwommen oder
  pixelig. Bilder aus dem Internet eignen sich in der Regel nicht zum Druck.}

Auch die Qualität Ihrer Bilder ist wichtig. Die Größe einer Bilddatei gibt
dabei meist einen Hinweis auf die Druckqualität (Auflösung). Falls Sie die
Bildauflösung nicht ermitteln können, machen Sie am besten einen Ausdruck der
betreffenden Seite in Originalgröße, um das Ergebnis zu überprüfen.

\paragraph{Druck}\strut\\
\marginpar{\RaggedRight\footnotesize Die kleine sogenannte Marginalspalte
  außen liefert Zusatzinformationen oder Randbemerkungen zum Haupttext
  nebenan.}
Am besten lassen Sie mehrseitige Broschüren professionell drucken, denn je
nach Art der Bindung und Seitenanzahl ist eine bestimmte Anordnung der Seiten
(Ausschießen) notwendig. Wenn darüber hinaus Bilder eine Seite vollflächig
(randabfallend) aus füllen, schneidet der Drucker die Seiten so ab
(Beschnitt), dass es keinen weißen Rand gibt.\par

Broschüren haben mindestens 8 Seiten (das wären 4 Seiten Umschlag und 4
Innenseiten) und wachsen immer um 4 Seiten (8, 12, 16, 20 etc.). Denn das
entspricht einem weiteren Papierbogen.\par

Je nach Seitenanzahl, Papierwahl und Verwendung gibt es unterschiedliche
Möglichkeiten der Bindung. Zwei übliche Verfahren wären:
\begin{itemize}
  \item Rückendrahtheftung – eine einfache Bindung mit Klammern
    (Draht) im Falz wie bei vielen Magazinen.
  \item Klebebindung – man arbeitet hierfür mit unterschiedlichen Verfahren
    und Klebemitteln. Generell werden die Innenseiten am Rücken verklebt und
    mit einem Umschlag versehen wie beim Forschungsmagazin oder bei
    Taschenbüchern. In dieser Variante sind ungerade Seitenzahlen möglich.
\end{itemize}
Wenn Sie Ihr Dokument z. B. zur Überprüfung selbst ausdrucken möchten,
empfiehlt es sich, ein PDF in Originalgröße zu erstellen. Achten Sie immer
darauf, dass das Dokument in Originalgröße gedruckt wird, also keine Anpassung
an die Druckränder bei Ihren Druckereinstellungen aktiviert ist.\par

Wählen Sie bitte unbedingt weiße Papiere statt Naturpapier mit starker Braun-
oder Graufärbung. Denn nur sie entsprechen dem Charakter des Corporate Designs
der TUM. Denken Sie daran, dass auch Papierstärke (Dicke) und Haptik einen
Einfluss auf das Druckergebnis haben.\par

Für den doppelseitigen Druck ist im Zweifelsfall die Nutzeranweisung des
jeweiligen Druck-Gerätes zu beachten. Beachten Sie dass Broschürendruck nicht
mit allen Programmversionen und Druckern möglich ist.\par

Weitere Informationen zum Corporate Design der TUM finden
Sie unter \url{www.tum.de/cd}

% .:: Page 5, 6, and 7
% ----------------------------------------------------------------------------
\clearpage
\pagestyle{TUM.insidepage}
{\usekomafont{TUM.H1} Beispiele für Bildeinsatz\par}

\begin{figure}[h]
  \begin{subfigure}[b]{0.49\textwidth}
    \includegraphics[width=\textwidth, height=.35\textheight]{example-image-a}
    \caption{Hier steht eine Bildunterschrift.}
  \end{subfigure}
  \hfill
  \begin{subfigure}[b]{0.49\textwidth}
    \includegraphics[width=\textwidth, height=.35\textheight]{example-image-b}
    \caption{Hier steht eine Bildunterschrift.}
  \end{subfigure}
  \vskip\baselineskip
  \begin{subfigure}[b]{0.49\textwidth}
    \includegraphics[width=\textwidth, height=.35\textheight]{example-image-c}
    \caption{Hier steht eine Bildunterschrift.}
  \end{subfigure}
  \hfill
  \begin{subfigure}[b]{0.49\textwidth}
    \includegraphics[width=\textwidth, height=.35\textheight]{example-image}
    \caption{Hier steht eine Bildunterschrift.}
  \end{subfigure}
\end{figure}

\clearpage
{\usekomafont{TUM.H1} Beispiele für Bildeinsatz\par}

\begingroup
\includegraphics[width=\textwidth, height=0.85\textheight]{example-image}
\captionsetup{format=hang}
\captionof{figure}{Link Seite: Großes Bild im vorgesehenen
  Gestaltungsrahmen -- mit Bildunterschrift.\\
  Rechte Seite: Großes randabfallendes Bild -- Ihre Bildunterschrift muss z.B.
  im Abbildungsverzeichnis stehen.}
\endgroup

\newgeometry{width=\paperwidth, height=\paperheight}
\includegraphics[width=0.999\paperwidth,
  height=0.999\paperheight]{example-image-a}


% .:: Page 8
% ----------------------------------------------------------------------------
\pagestyle{TUM.titlepage}
\strut\vfill
\theUniversityName\\
\theDepartmentName\\
\theChairName\\
\\
Arcisstraße 21\\
80333 München\\
\\
\textbf{www.fk.tum.de}
\end{document}
