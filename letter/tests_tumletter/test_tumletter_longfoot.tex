\documentclass[german]{tumletter}


% set letter specific variables (overwriting defaults)
\setkomavar{fromname}{Erika Mustermann}
\setkomavar{fromaddress}{Arcisstraße 21\\80333 München}
\setkomavar{backaddress2}{2. Adresszeile optional\\löschen falls nicht nötig}
\setkomavar{date}{1. März 2021}

% chair and department information (overwriting defaults from tumuser.sty)
\provideName{\theDepartmentName}{Some clinic}{Musterklinikum}
\provideName{\theChairName}{Any Group}{Abteilung für Musterverfahren}

% letter subject
\setkomavar{subject}{Betreffzeile}


% modify footer column 2
\setkomavar{tum.footer.col2}{%
  \textbf{Prof.\@~Dr.\@ Muster}\\Ärztlicher Direktor\\
  \vspace{\baselineskip} % add one empty line
  \textbf{Muster Direktor}\\Kaufmännischer Direktor\\
  \vspace{\baselineskip} % add one empty line
  \textbf{Muster Direktor}\\Pflegedirektorin\\
  \vspace{\baselineskip} % add one empty line
  \textbf{Prof.\@~Dr.\@ Muster Arzt}\\Dekan
}

% modify footer column 3
\setkomavar{tum.footer.col3}{%
  Musterabteilung\\Musterstraße 22\\81234 Musterstadt\\
  \TUMEmail{muster.direktor@tum.de}\\\TUMPhoneNumbers[12346]{12345}\\
  \vspace{\baselineskip} % add one empty line
  Musterklinikum\\Musterweg 1\\80333 München\\
  \TUMEmail{muster.arzt@tum.de}\\\TUMPhoneNumbers[11112]{11111}
}


\begin{document}
\begin{letter}{Fa. Muster Mechanik\\Herrn Max Mustermann\\
    Musterstraße 12\\80333 München}

  \opening{Sehr geehrter Herr Mustermann,}
  \raggedright

  dies ist die Vorlage für den Brief der Technischen Universität München
  (TUM). Die Vorlage ist nach dem aktuellen Corporate Design der TUM gestaltet
  und verbindlich. Zur Optimierung für den C4-Umschlag mit Fenster fügen Sie
  bitte eine Leerzeile vor Ort und Datum ein.

  Bitte geben Sie Ihren individuellen Text an den vorgesehenen Stellen ein und
  nutzen Sie die installierten Formatvorlagen für den jeweiligen Absender. Die
  Fußzeile wächst mit, falls weitere Informationen untergebracht werden
  müssen. Wir empfehlen nicht mehr als 15 Zeilen. Bitte geben Sie stets
  Fakultät und Lehrstuhl an und achten Sie bei Ihrem Namen darauf den
  vollständigen akademischen Grad anzugeben. In der Kopfzeile steht nur das
  Logo der TUM in der Markenfarbe.

  Achten Sie ebenfalls darauf, dass das Dokument in Originalgröße gedruckt
  wird, also keine Anpassung an die Druckränder bei Ihren Druckereinstellungen
  aktiviert ist, damit die Falz- und Lochmarken an der richtigen Stelle
  stehen. Weitere Informationen zum Corporate Design der TUM finden Sie unter
  \TUMWebsite{www.tum.de/cd}.

  \closing{Mit freundlichen Grüßen}
\end{letter}
\end{document}
